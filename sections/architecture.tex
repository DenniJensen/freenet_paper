\section{Architektur}
Ein Freenet Netzwerk besteht aus Nutzer Knoten, die ihren lokalen Speicher
anderen Nutzern zur Verfügung stellen. Jeder Knoten hat einen direkten Nachbarn
von dem gelesen wird und einen direkten Nachbarn zu dem Dateien verteilt
werden. Des Weiteren verwaltet jeder Knoten eine dynamische Routingtabelle.
Jeder Routingtabelle beinhaltet die Adressen von Knoten und deren Schlüssel.
Genutzt wird Freenet um die eigene Speicherkapazität zu erhöhen und um
ungenutztem Speicher zur Verfügung zu stellen.

Anfrage von Schlüsseln von Knoten zu Knoten erfolgt über eine Kette von
Proxy-Requests, indem jeder Knoten entscheiden an wem der nächste Request geht.
Dieses Verfahren ähnelt dem IP-Routing. Jede Route ändert sich abhängig vom
angefragtem Schlüssel.
Um die Privatsphäre von Knoten zu unterstützen haben Knoten nur Informationen
über den ummittelbaren Nachbarn(up- und downstream).

Jeder Request erhält einen hops-to-live Wert, der sich bei jedem angelangtem
Knoten verringen. Dadurch werden Endlosschleifen vermieden.
Jeder Request enthält eine Psuedo-uniq Zufalls-ID, um Endlossschleifen zu
verhindern. Jeder Request, der zum zweiten mal bei einem Knoten ankommt, fliegt
raus. Es gibt keine Herachie der Knoten. Alle Knoten sind gleichgestellt
im Freent.

\subsection{Schlüssel}
Dateien im Freenet werden mit einem 160 Bit SHA-1 Schlüssel dargestellt.
Freenet hat 3 verscheidene Schlüssel, die einen bestimmten Teil einer Datei
beschreiben.

\subsubsection{KSK - Keyword-Signed-Key}
Der Keyword-Signed-Key dient für die Beschreibung einer Datei im Freenet. Jeder
Benutzer der eine Datei anlegt, wählt einen Text zur Idenfizierung diese
Datei. Wie z.B.
\begin{lstlisting}
text/philosophy/sun-trz/art-of-war
\end{lstlisting}
Aus diesem Text wird deterministisch ein public-private Schlüsselpaar
generiert. Der Public Schlüssel wird anschließend gehasht um den Dateischlüssel
zu erhalten, der dann mit dem privatem Schlüssel signiert wird.  Der
Dateischlüssel ist der Schlüssel mit der die Datei dargestellt wird. Dieser
Schlüssel kann mit dem gewähltem Text erneut berechnet und gesucht werden.  Um
anderen Nutzern den Zugriff auf die Datei zu gewähren, verschickt der Author
der Datei den beschreibenden String. Keyword-Signed-Key sind so leicht zu
merken.  Ein Problem bei dem Keyword-signed-Key ist der geringe Namespace,
welcher dazu frühren kann, das unterschiedliche Datei die selbe Bezeichnung und
dadruch den selben Dateischlüssel erhalten. Um das Problem mit der
Nameskollidierung zu umgehen dient der Signed-Subsapce-Key

\subsubsection{SSK - Signed-Subsapce-Key}
Der Signed-Subspace-Key ist eine Erweiterung des KSK. SSK erweitert den
Namespace und reduziert Kollisionen, wo zwei verschieden Dateien den
selben Schlüssel erhalten können.
Ein zufällig gewähltes assymetrisches Schlüsselpaar stellt den Namespace dar.
Um eine Datei hinzuzufügen wählt der Nutzer ein beschreibenden Text. Der
public Signed-Subsapce-Key und der beschreibende Text, werden unabhängig
voneinander gehasht, anschließend miteinander geXOR't und dann wieder gehasht.

\subsubsection{CHK - Content-Hash-Key}
Der Content-Hash-Key representiert den Inhalt der Datei in Form eines Hashes.
Jede Datei erhält dadurch einen pseudo einzigartigen Schlüssel. Zusätzlich wird
eine Datei mit einem zufällig generiertem Schlüssel verschlüsselt.
Um Zugriff auf die Datei zu ermöglichen, veröffentlicht der Nutzer den
Content-Hash-Key und den dazu gehörigen Schlüssel zur Entschlüsselung.

Eine Datei kann kann mit dem Content-Hash-Key durch den Author aktualisiert
werden. Dazu wird ein neuer Content-Hash-Key erzeugt und unter dem
Signed-Subspace-Key abgelegt. Der neue Schlüssel muss sich vom Alten
unterscheiden. Sollte die neue Datei einen Knoten ereichen, der noch auf die
alte Version verweist, ensteht eine Kollision, die dazu führt, dass der Knoten
die Signatur der neuen Version überprüft, mit der alten Version validiert und
auf die neue Version verweißt. Der Signed-Subspace-Key
verweist immer auf die aktuellste Version der Datei. Die alte Version kann
weiterhin erreicht werden, indem direkt der Content-Hash-Key der älteren
Versioin angefragt wird.  Wird die ältere Version über einen bestimmten
Zeitraum nicht mehr aufgerufen, kann diese vom Freenet entfernt werden.

Eine weiter Funktion vom Content-Hash-Key ist das Aufteilen einer Datei
in mehrere Teildatei, um z.B. große Daten mit weniger Bandbreite über das
Netzwerk zu verschicken.
