\section{Sicherheit}
Freenet hat das Ziel die Anonymität der Autoren und Leser von Dateien zu
schützen. Selbst die Dateien müssen im Freenet vor feindlichen Modifikationen
geschützt werden und das Freenet-System muss resistent gegen DOS-Angriffen
sein.

Freenet stützt sich auf drei wesentliche Aspekte, die aus der Ausarbeitung
von Reiter und Rubin stammen \cite{reiterandrubin}. Der erste Aspekt ist der
Punkt der Anonymität. Es soll nicht ermittelt werden können wer eine Nachricht
im Netz abgeschickt hat oder an wem diese Nachricht verschickt wurde.
Der Aspekt ist die Erkennung von potentiellen Feinden im Freenet. Zu den
potenziellen Angreifern gehören, lokale Lauscher, bösartige Knoten oder eine
Zusammenarbeit von bösartigen Knoten. Die dritte Achse ist der Grad der
Anonymität.

Im Freenet können die angefragten Schlüssel nicht versteckt werden, da
sie ein Bestandteil des Routingmechanismuses sind und jede Routing-Tabelle auf
die Schlüssel angewiesen ist. Dadurch ist eine Anonymität der Schlüssel nicht
gewährleistet. Der Nutzen von Hashes als Schlüssel stellt dennoch eine hohe
Unklarheit gegen gelegentlichen Mithörern auf.
Gegen Wörterbuchangriffen bleibt sind die Schlüssel dennoch anfällig.

Die Anonymität von Sender-Knoten in einem Bund aus böswilligen Knoten wird
gewährleistet, da ein Knoten der eine Nachricht erhält, nicht bestimmen kann
ob der Vorgänger die Nachricht initialisierte oder einfach nur
weiterleitete. Durch das zufällige Setzen des Depth-Counters wird das
Zurückverfolgen von Requests drastisch erschwert.

Es gibt keinen Schutz gegenüber lokalen Mithörern, die Nachrichten vom Nutzer
bis zum ersten Knoten beobachten. Solange der erste Knoten ein mithörender
Knoten sein könnte, wird es empfohlen, dass den ersten Knoten für
den Einstieg ins Freenet von der eigenen Maschine zu wählen. Nachrichten
zwischen Knoten sind verschlüsselt und können nicht mehr abgehört werden. Die
einzige Beobachtung bleibt der Ausgang einer Nachricht den Eingang einer
Anfrage, was dazu schließen lass kann, das der Knoten der Absender sein könnte.

Datenquellen können geschützt werden indem die Felder der Datenquellen
regelmäßig zurückgesetzt werden.
Im Freenet ist es nicht möglich die Quelle einer Datei zu ermitteln. Liefert
der Downstream-Knoten eine Datei, kann diese von dem Knoten selbst stammen oder
von einem anderen Knoten, die an den Downstream-Knoten gesendet wurde.

Daten, die unter den Content-Hash oder den Signed-Subspace Schlüsseln
gespeichert werden können nicht manipuliert werden, da unechte Dateien im
Freenet nachgewiesen werden können. Die einzige Möglichkeit den Inhalt einer
Datei zu manipulieren besteht, wenn der Angreifer eine Kollision verursacht und
die Signatur fälschen könnte. Unter den Keyword-Signed Schlüssel gespeicherte
Dateien sind anfällig gegen Wörterbuchangriffe.

Weitere Angriffe sind die Denial-Of-Service Attacken. Ein Angreifer könnten
versuchen mit einer hohen Anzahl an Abfalldateien das Freenet-Netzwerk zu
fluten. Eine möglicher Konter ist es den Nutzer lange Berechnung vor der
Einfügung durchzuführen zu lassen. Damit würde das Netzwerk entlastet werden.
Ein weiter Methode ist die Aufteilung des Speicher in zwei Segmente.
Ein Speicher dient für neue Dateien und ein Speicher für Dateien, die eine
gewisse Anzahl an Anfragen aufweisen. Mit dieser Methode würden die neuen Datein
nur neue Dateien ersetzen können. Dadurch würden keine Berechnung auf die
etablierten Dateien ausgeführt werden. Die Fluten wird dadurch gestoppt, dass
wenn ein Knoten eine Anfrage versendet, die vom ersten Knoten schon gestoppt
wird, da der Knoten die Datei beinhalten. Eine Verbreitung über das Netzwerk
ist dadurch verhindert.

Ein Ziel von Angreifern kann es sein existieren Dateien im Freenet zu ersetzen.
Wie bei den DOS-Attacken ist ein Angriff auf die Content-Hash oder
Signed-Subspace Schlüssel nicht möglich.
