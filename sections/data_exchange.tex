\section{Datenaustausch}
\subsection{Daten empfangen}
Um Daten im Freenet zu empfangen, wählt oder berechnet der Nutzer einen Hash.
Dieser Hash wird dann als Request-Nachricht über eine Kette von Requests und
mit einer gegebenen hops-to-live-Zahl versendet. Als Start dient der eigene Knoten.
Der Knoten, bei dem die Request-Nachricht ankommt, überprüft, ob der gesuchte
Hash vorhanden ist. Sollte die Datei gefunden werden, sendet der Knoten
die Datei zurück. Sollte der gesuchte Schlüssel nicht vorhanden sein, wird
der nächste Knoten aus der Routingtabelle genommen, um die Request-Nachricht
weiterzuleiten. Dabei wird die hops-to-live-Zahl um eins verringert. Dieser
Vorgang wiederholt sich so lange, bis der hops-to-live-Wert auf null fällt.
Kann ein Knoten einen Request nicht mehr weiterleiten, wird der vorherige
Knoten benachrichtigt. Dieser sucht sich einen neuen Knoten aus der
Routingtabelle und versendet den Request neu. Sollte der hops-to-live-Wert auf
null fallen und es wurde keine Datei mit dem Schlüssel gefunden, wird eine
Fehlernachricht bis zum Anfang der Kette weitergeleitet.
Wird die Datei gefunden, wird dieser Hash in allen Knoten eingetragen, die
Bestandteil der Request-Kette waren (TODO Aussage nochmal prüfen).

\subsection{Daten verteilen}
Verteilen von Daten ist ähnlich wie das Empfangen von Daten. Der Nutzer
berechnet den Hashwert für die Datei und versendet diese an den eigenen Knoten.
Dieser überpüft, ob der Hashwert schon vorhanden ist. Sollte das der Fall sein,
kommt es zur Kollision und der Nutzer muss einen neuen Schlüssel berechnen.
Sollte der Schlüssel nicht vorhanden sein, leitet der Knoten den Request
weiter, bis der hop-to-live-Wert aufgebraucht ist oder die Datei gefunden
wurde. Sollte die Datei gefunden werden, kommt es wieder zur Kollision und die
Fehlernachricht wird bis zum Initialknoten durchgereicht, der dann wieder einen
neuen Schlüssel berechnen muss. Sollte die Datei nicht gefunden werden, wird
die Datei in allen Knoten eingefügt, die an der Request-Kette beteiligt waren.
Sollte ein Knoten keinen Speicherplatz für die neue Datei verfügbar haben,
wird die am wenigsten aufgerufene Datei ersetzt. Freenet garantiert somit keine
permanente Persistierung von Daten sicher.
