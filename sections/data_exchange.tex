\section{Daten Austausch}
\subsection{Daten Empfangen}
Um Daten im Freenet zu empfangen, wählt oder berechnet der Nutzer einen Hash.
Dieser Hash wird dann als Request Nachricht über eine Kette von Request
versendet mit dem Start des eigenen Knotens und einer gegebenen hops-to-live.
Der Knoten, bei dem die Request-Nachricht ankommt, überprüft ob der gesuchte
Hash schon vorhanden ist. Sollte die Datei gefunden werden, sendet der Knoten
die Datei zurück. Sollte der gesuchte Schlüssel nicht vorhanden sein, schaut
der Knoten in der Routingtabelle nach einem nächsten Knoten, um den Request
weiterzuleiten und verringert die hops-to-live um eins.
Dieser Vorgang wiederholt sich solange bis der hops-to-live Wert auf null
gefallen ist.
Kann ein Knoten einen Request nicht mehr weiterleiten, dann wird der vorherige
Knoten benachrichtigt und dieser sucht sich einen neuen Knoten aus der
Routingtabelle und versendet den Request neu.
Sollte der hops-to-live Wert auf null fallen und es wurde keine Datei mit dem
Schlüssel gefunden, wird ein Fehlernachricht bis zum Anfang der Kette
versendet.

\subsection{Daten Verteilen}
Verteilen von Daten ist ähnlich, wie das Empfangen von Daten. Der Nutzer
berechnet den Hashwert für die Datei und versendet diese an den eigenen Knoten.
Dieser überpüft, ob der Knoten diesen Hashwert beinhalten. Sollte das nicht
der Fall sein, leitet es den Request weiter, bis der hop-to-live Wert
aufgebraucht ist oder die Datei gefunden wurde. Sollte die Datei gefunden
werden, bekommt der Requester die Fehlernachricht, dass es der Schlüssel schon
vorhanden ist. Sollte die Datei nicht gefunden werden, wird die Datei in
allen Knoten eingefügt, wo sie nachgefragt wurde. Sollte ein Knoten keine
Speicherplatz für die neue Datei zu Verfügung haben, wird die Datei ersetzt,
die am wenigsten abgerufen wurde.
Freent stellt somit kein permanente Persitierung von Daten sicher.

