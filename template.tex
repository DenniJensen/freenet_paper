\documentclass[paper=a4, fontsize=11pt, twocolumn]{scrartcl}

\input{preamble.tex}

%%% Title
\title{ \vspace{-1in} 	\usefont{OT1}{bch}{b}{n}
		\huge \strut  Seminar: Peer-To-Peer Overlay Network Systems S15\strut \\
		\Large \bfseries \strut Freenet: A Distribyted Anonymous Information Storage and Retrieval System \strut
}

\author{ 					\usefont{OT1}{bch}{m}{n}
  Dennis Hägler\\		\usefont{OT1}{bch}{m}{n}
  Freie Universität Berlin\\	\usefont{OT1}{bch}{m}{n}
  \texttt{dennis.haegler@fu-berlin.de}
}

%%% Begin document
\begin{document}
\twocolumn[
\begin{@twocolumnfalse}
\maketitle
\begin{abstract}
Freenet ist ein Peer-To-Peer Overlay Netzwerk und dient für den anonymen
Austausch von Dateien. Die Funktionweise von Freenet, die Umsetzung der
Anonymität im Freenet, sowie der Aufbau von Freenet werden in dieser
Ausarbeitung aufgeführt. Als Vorlage diente das Paper \cite{freenetpaper}.
\vspace{4em}
\end{abstract}
\end{@twocolumnfalse}
]

\section{Einleitung}
Das Ziel von Freenet besteht den anonymen Austausch von Informationen zu
ermöglichen und abzuspeichern.  Dieses Ziel soll durch Dezentralisierung,
Redundanz, Verschlüsselung und dynamisches Routing erreicht werden.  Es wurde
als freie Software unter der GNU General Public License entwickelt.

\section{Architektur}
Ein Freenet Netzwerk besteht aus Nutzer, die ihren lokalen Speicher anderen
Nutzern zur Verfügung stellen. Jeder Knoten hat einen direkten Nachbarn von dem
gelesen wird und einen direkten Nachbarn zu dem Dateien verteilt werden.
Des Weiteren verwaltet jeder Knoten eine dynamische Routingtabelle. Jeder
Routingtabelle beinhaltet die Adressen von Knoten und deren Schlüssel.
Genutzt wird Freenet um die eigene Speicherkapazität zu erhöhen und um
ungenutztem Speicher zur Verfügung zu stellen.

Anfrage von Schlüsseln von Knoten zu Knoten erfolgt über eine Kette von
Proxy-Requests, indem jeder Knoten entscheiden an wem der nächste Request.
Dieses Verfahren ähnelt dem IP-Routing. Jede Route ändert sich abhängig vom
angefragtem Schlüssel.
Um die Privatsphäre von Knoten zu unterstützen haben Knoten nur Informationen
über den ummittelbaren Nachbarn(up- und downstream).

Jeder Request erhält einen hops-to-live Wert, der sich bei jedem angelangtem
Knoten verringen. Dadurch werden Endlosschleifen vermieden.

\subsection{Schlüssel}
Dateien im Freenet werden mit einem 160 Bit SHA-1 Schlüssel dargestellt.
Freenet hat 3 verscheidene Schlüssel, die einen bestimmten Teil einer Datei
beschreiben.

\subsubsection{KSK - Keyword-Signed-Key}
Der Keyword-Signed-Key dient für die Beschreibung einer Datei im Freenet. Jeder
Benutzer der eine Datei anlegt, wählt einen Text zur Idenfizierung diese
Datei. Wie z.B.
\begin{lstlisting}
text/philosophy/sun-trz/art-of-war
\end{lstlisting}
Aus diesem Text wird deterministisch ein public-private Schlüsselpaar
generiert. Der Public Schlüssel wird anschließend gehasht um den Dateischlüssel
zu erhalten, der dann mit dem privatem Schlüssel signiert wird.  Der
Dateischlüssel ist der Schlüssel mit der die Datei dargestellt wird. Dieser
Schlüssel kann mit dem gewähltem Text erneut berechnet und gesucht werden.  Um
anderen Nutzern den Zugriff auf die Datei zu gewähren, verschickt der Author
der Datei den beschreibenden String. Keyword-Signed-Key sind so leicht zu
merken.  Ein Problem bei dem Keyword-signed-Key ist der geringe Namespace,
welcher dazu frühren kann, das unterschiedliche Datei die selbe Bezeichnung und
dadruch den selben Dateischlüssel erhalten. Um das Problem mit der
Nameskollidierung zu umgehen dient der Signed-Subsapce-Key

\subsubsection{SSK - Signed-Subsapce-Key}
Der Signed-Subspace-Key ist eine Erweiterung des KSK. SSK erweitert den
Namespace und reduziert Kollisionen, wo zwei verschieden Dateien den
selben Schlüssel erhalten können.
Ein zufällig gewähltes assymetrisches Schlüsselpaar stellt den Namespace dar.
Um eine Datei hinzuzufügen wählt der Nutzer ein beschreibenden Text. Der
public Signed-Subsapce-Key und der beschreibende Text, werden unabhängig
voneinander gehasht, anschließend miteinander geXOR't und dann wieder gehasht.

\subsubsection{CHK - Content-Hash-Key}
Der Content-Hash-Key hasht den Inhalt der Datei und representiert den Inhalt
anhand des Hashes. Jede Datei erhält dadurch einen pseudo einzigartigen
Schlüssel. Zusätzlich wird eine Datei mit einem zufällig generiertem Schlüssel
verschlüsselt.
Um Zugriff auf die Datei zu vergeben, veröffentlich der Nutzer den
Content-Hash-Key und den beschreibendem Text.

\section{Datenaustausch}
\subsection{Daten empfangen}
Um Daten im Freenet zu empfangen, wählt oder berechnet der Nutzer einen Hash.
Dieser Hash wird dann als Request-Nachricht über eine Kette von Requests und
mit einer gegebenen hops-to-live-Zahl versendet. Als Start dient der eigene Knoten.
Der Knoten, bei dem die Request-Nachricht ankommt, überprüft, ob der gesuchte
Hash vorhanden ist. Sollte die Datei gefunden werden, sendet der Knoten
die Datei zurück. Sollte der gesuchte Schlüssel nicht vorhanden sein, wird
der nächste Knoten aus der Routingtabelle genommen, um die Request-Nachricht
weiterzuleiten. Dabei wird die hops-to-live-Zahl um eins verringert. Dieser
Vorgang wiederholt sich so lange, bis der hops-to-live-Wert auf null fällt.
Kann ein Knoten einen Request nicht mehr weiterleiten, wird der vorherige
Knoten benachrichtigt. Dieser sucht sich einen neuen Knoten aus der
Routingtabelle und versendet den Request neu. Sollte der hops-to-live-Wert auf
null fallen und es wurde keine Datei mit dem Schlüssel gefunden, wird eine
Fehlernachricht bis zum Anfang der Kette weitergeleitet.
Wird die Datei gefunden, wird dieser Hash in allen Knoten eingetragen, die
Bestandteil der Request-Kette waren (TODO Aussage nochmal prüfen).

\subsection{Daten verteilen}
Verteilen von Daten ist ähnlich wie das Empfangen von Daten. Der Nutzer
berechnet den Hashwert für die Datei und versendet diese an den eigenen Knoten.
Dieser überpüft, ob der Hashwert schon vorhanden ist. Sollte das der Fall sein,
kommt es zur Kollision und der Nutzer muss einen neuen Schlüssel berechnen.
Sollte der Schlüssel nicht vorhanden sein, leitet der Knoten den Request
weiter, bis der hop-to-live-Wert aufgebraucht ist oder die Datei gefunden
wurde. Sollte die Datei gefunden werden, kommt es wieder zur Kollision und die
Fehlernachricht wird bis zum Initialknoten durchgereicht, der dann wieder einen
neuen Schlüssel berechnen muss. Sollte die Datei nicht gefunden werden, wird
die Datei in allen Knoten eingefügt, die an der Request-Kette beteiligt waren.
Sollte ein Knoten keinen Speicherplatz für die neue Datei verfügbar haben,
wird die am wenigsten aufgerufene Datei ersetzt. Freenet garantiert somit keine
permanente Persistierung von Daten sicher.

\input{sections/protocol_details}
\input{sections/security}
\bibliography{bibliography}
\end{document}
